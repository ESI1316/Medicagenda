\documentclass[a4paper, 11pt]{report}

\usepackage[utf8]{inputenc}                                                      
\usepackage[french]{babel}                                                       
\usepackage[T1]{fontenc}  
\usepackage[pdftex]{graphicx}                                                    
\usepackage{url}                                                                 
\usepackage{longtable}
\usepackage[bookmarks, colorlinks=false, pdfborder={0 0 0},
	pdftitle={Medicagenda}, 
	pdfauthor={Knop Florian, Kriwin Paul, Placentino Simon},
	pdfsubject={Medicagenda},           
pdfkeywords={UML, ISO/IEC 19505-1:2012, Medicagenda, Projet, analyse, ESI}]{hyperref}   

\newcommand{\HRule}{\rule{\linewidth}{0.5mm}}    

\begin{document}
\begin{titlepage}
	\begin{center}

		\includegraphics[keepaspectratio=true,width=0.20\textwidth]{../ressources/logo}\\[1cm]

		\textsc{\LARGE H.E.B. Ecole Superieur d'Informatique}\\[1.5cm]

		\textsc{\Large Laboratoire d'analyse : Projet d'analyse}\\[0.5cm]
		\textsc{\Large Modèle Conceptuel des Traitements}\\[0.5cm]

		\HRule \\[0.4cm]
		{\huge \bfseries Medicagenda \\[0.4cm]}
		\HRule \\[1.5cm]

		\noindent
		\begin{minipage}[t]{0.4\textwidth}
			\begin{flushleft} \large
				\emph{Auteurs:}\\
				Florian \textsc{Knop} \href{mailto:39310@heb.be}{39310@heb.be}\\
				Paul \textsc{Kriwin} \href{mailto:39171@heb.be}{39171@heb.be}\\
				Simon \textsc{Placentino} \href{mailto:39631@heb.be}{39631@heb.be}\
			\end{flushleft}
		\end{minipage}%
		\begin{minipage}[t]{0.4\textwidth}
			\begin{flushright} \large
				\emph{Titulaire du cours:} \\
				Mr.~Nicolas \textsc{Pettiaux}
				\href{mailto:npettiaux@heb.be}{npettiaux@heb.be}
			\end{flushright}
		\end{minipage}

		\vfill

		{\large \today}

	\end{center}
	\clearpage\null\newpage
\end{titlepage}

\tableofcontents

\chapter{Modèle conceptuel des traitements}

\section{Introduction}

\subsection{Objectifs du document}
Ce document présente la décomposition fonctionnelle du projet Medicagenda.
\subsection{Domaine de définition du document}
L'ensemble des sous-systèmes décrits dans le MCD :
\begin{itemize}
	\item SS1 : Nom complet
	\item SS2 : Nom complet 
\end{itemize}
\subsection{Définitions, acronymes et abréviations}
\subsection{Références}
\begin{itemize}
	\item[] Etude de cas de l'agenda médical\footnote{path{../analyse.pdf}}
	\item[] M.C.D. de Medicagenda
\end{itemize}
\newpage
\section{Acteurs}
\subsection{Acteurs externes}
Les acteurs externes au SIA complet du Medicagenda se retrouvent dans le 
diagramme de contexte\footnote{Voir~\ref{dc} Diagramme de contexte} : 
\begin{itemize}
	\item Patient : Personne qui pourra prendre rendez-vous auprès d'un ou
		plusieurs médécin pour une ou plusieurs spécialisation.
\end{itemize}
\subsection{Acteurs internes}
\begin{itemize}
	\item Médecin : Propriétaire de l'agenda pouvant éventuellement être le
		gestionnaire de ce dernier.
\end{itemize}
\subsection{Diagramme des acteurs du SIA}
\begin{figure}[hb]
	\centering
	\includegraphics[scale=0.7]{acteurs.jpg}
	\caption{Diagramme des acteurs du SIA}
	\label{fig:acteurs}
\end{figure}
\newpage
\section{Vision globale du SI}
\subsection{\label{dc}Diagramme de contexte}
Ce diagramme présente tout le système informatique du Medicagenda et les acteurs 
externes avec lesquels il est en relation. 
\begin{figure}[hb]
	\centering
	\includegraphics[scale=0.7]{contexte.jpg}
	\caption{Diagramme de contexte}
	\label{fig:contexte}
\end{figure}
\subsection{Diagramme des systèmes}
\begin{figure}[hb]
	\centering
	\includegraphics[scale=0.7]{systemes.jpg}
	\caption{Diagramme des systèmes}
	\label{fig:systemes}
\end{figure}
\newpage
\section{Diagramme des sous-sytèmes}
\begin{figure}[hb]
	\centering
	\includegraphics[scale=0.7]{sous-systeme.jpg}
	\caption{Diagramme des sous-systèmes}
	\label{fig:sous-systeme}
\end{figure}
\section{Système SI1 : Agenda}
\subsection{Diagramme des Use Cases}
\subsection{Description des U.C.}
\subsubsection{UC XXXX}

\newpage
\subsection{Matrice CRUD}

Pour la lisibilité de la matrice, cette dernière a été séparé en deux tableaux.
Seuls les UC ayant une action avec une des classes du tableau apparaissent.

\begin{center}
    \begin{longtable}{|p{1.5cm}|p{1.5cm}|p{1.5cm}|p{1.5cm}|p{1.5cm}|}
        \hline
        & Compte & Compte Mèdecin & Compte Patient & Spécialité \\
        \hline
        UC1101a & C & / & C & / \\
        \hline
        UC1101b & C & C & / & R \\
        \hline
        UC1102  & U & / & U & / \\
        \hline
        UC1105  & R & R & R & R \\
        \hline
        UC1106 & R & R & R & R \\
        \hline
        
    \end{longtable}
\end{center}

\begin{center}
    \begin{longtable}{|p{2.2cm}|p{2.2cm}|p{2.2cm}|p{2.2cm}|}
        \hline
        & Evènement & Rendez-vous & Disponnibilité  \\
        \hline
        UC1103 & C & R & C \\
        \hline 
        UC1104 & C & C & R \\
        \hline
        UC1105 & U & U & / \\
        \hline
        UC1106 & R & R & / \\
        \hline
        
    \end{longtable}
\end{center}


\end{document}
