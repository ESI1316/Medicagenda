\chapter{Plan de test fonctionnels élémentaires \texttt{U.C. 1106}}
\section{Introduction}
\subsection{Objectifs du document}
Ce document présente le plan des tests fonctionnels du \texttt{U.C. 1106} -
envoyer une notification. 
\subsection{Domaine de définition du document}
Ce document ne présente que les scénarios correspondant à l'envoi d’une
notification (correspondant au diagramme d’activités de la description 
fonctionnelle du \texttt{U.C.}).
\subsection{Définitions, acronymes et abréviations}
Définition INAMI : voir étude de cas et MCD\footnote{\href{../MCD/MCD.pdf}{voir document M.C.D.}}.
\subsection{Références}
\begin{itemize}
	\item[] Étude de cas de l'agenda
		médical\footnote{\href{../Enonce_Travail_Synthese_14-15.pdf}{voir
		étude de cas}}
	\item[] M.C.D. de \texttt{Medicagenda}\footnote{\href{../MCD/MCD.pdf}{voir document M.C.D.}}
	\item[] M.C.T. de \texttt{Medicagenda}\footnote{\href{./MCT.pdf}{voir document M.C.T.}}
	\item[] Spécifications fonctionnelles du \texttt{U.C.
		1106}\footnote{\href{./specifications_fonctionnelles_UC_1106_envoyer_une_notification.pdf}{Voir
			spécifications fonctionnelles du \texttt{U.C. 1106}}}
	\end{itemize}

	\section{Types de tests élémentaires fonctionnels}
	\subsection{Validation des règles de saisie}
	Aucune car il n’existe pas d’interface dans un \texttt{U.C.} automatisé.
	\subsection{Validation des règles de calcul}
	\texttt{S.O.}
	\subsection{Validation des mises à jour du SI}
	Aucune, ce \texttt{U.C.} accède à la base de donnée uniquement lecture.
	\subsection{Validation des outputs utilisateurs}
	Validation du contenu et de l'affichage des messages de notification. 
	Ces derniers doivent être exempts d’erreurs au sein de leurs contenus et
	garantir qu'ils sont adressés au bon patient pour le bon rendez-vous.
	\subsection{Tests de consolidation}
	Tous les scénarios peuvent commencer par un encodage de données à 
	vérifier par l'interface.;
	\subsection{Tests de saisie}
	Aucuns, il s’agit d’un \texttt{U.C.} automatisé.
	\subsection{Cas extrêmes}
	Vérification des message en cas de relance du SI après une panne.
	\section{Scénarios de tests}
	\begin{itemize}
		\item[] \texttt{scénario 1. Cas 1.1}
			\begin{itemize}
				\item description:  \\
					Le patient possède un rendez-vous pour le lendemain, a fixé
					la notification à 18h la veille du rendez-vous et comme
					format de notification, l'email.
				\item résultat: \\
					A 18h doit être reçu sur la boîte email du patient, le
					message de notification indiquant le temps restant avant le
					rendez-vous et les informations inhérentes à celui-ci.
			\end{itemize}
		\item[] \texttt{scénario 1. Cas 1.2}
			\begin{itemize}
				\item description:  \\
					Idem que 1.1, avec la notification par SMS définie en plus. 
				\item résultat: \\
					Idem que 1.1 avec la réception au même moment d’un autre
					message de notification sur la boîte SMS du patient
			\end{itemize}
		\item[] \texttt{scénario 2. Cas 2.1}
			\begin{itemize}
				\item description:\\
					Idem que 1.2 sauf que quelques heures avant la notification
					prévue, le patient annule son rendez-vous.
				\item résultat: \\
					Le patient ne doit alors pas être notifié quand viendra
					l'heure de la notification du rendez-vous avorté. 
					Mais peu après avoir programmé l'annulation du rendez-vous,
					doit à la place, recevoir une notification de confirmation
					d’annulation.
			\end{itemize}
		\item[] \texttt{scénario 3. Cas 3.1}
			\begin{itemize}
				\item description:\\
					une panne du SI survient.
				\item résultat: \\
					Lors de la remise en marche de celui-ci, les notifications
					non envoyées à cause de la panne, doivent être complétées
					par un message d’excuse expliquant le problème technique
					survenu.
			\end{itemize}
	\end{itemize}
	\newpage
	\section{Valeurs}
	\subsection{Valeurs d'initialisation}
	Il faudra définir à chaque fois tous les enregistrements des tables de la
	base de données. 
	Ceci donnera lieu à des programmes de génération de données, et des 
	programmes permettant de faire évoluer les dates système.
	\subsection{Valeurs spécifiques}
	A définir au moment des tests.
