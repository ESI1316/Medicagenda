\chapter{Use Case Specification 1106 - Envoyer une notification}

\section{Introdcution}

\subsection{Objectifs de ce document}

Ce document présente les spécifications fonctionnelles du UC1106 - Envoyer une notification.

\subsection{Domaine de définition de ce document}

\section{Définition du Use Case}

\subsection{Identifiant et nom}

UC1106 - Envoyer une notification.

\subsection{Brève description}

Avant un rendez-vous, le système envoie un rappel au patient pour le prévenir d'un rendez-vous.
Le patient peut choisir le temps avant le rendez-vous pour envoyer le rappel ainsi que le moyen
de communication (e-mail, sms, etc.).

\section{Flux}

\subsection{Flux de base}

Cet UC est déclenché automatiquement un certain nombre d'heure avant un rendez-vous. Quand un patient ou un médecin
définissent un rendez-vous, un envoi de notification est programmé. L'heure est
définie grâce aux paramètres utilisateur, de même
pour le type d'envoi. Le patient peut annuler un rendez-vous, si le rendez-vous est annulé alors que la notification est déjà
programmée, la notification est annulée et une nouvelle notification confirmant
l'annulation est envoyée directement au 
patient.

\subsection{Flux alternatif}

\subsubsection{Interruption du UC}

Si le système subit une panne, il faudra contacter l'administrateur système.
Celui-ci lancera un UC pour relancer les notifications non envoyées avec un message d'excuse accompagnant ces
notifications.


\section{Acteurs, mode, etc.}

\subsection{Acteurs}

Le SIA est l'acteur principale, c'est à dire qu'en cas de panne, il faudra s'adresser au 
responsable du système.

\subsection{Mode}

Automatisé unitaire.

\subsection{Evènement déclencheur}

Après que le patient ou le médecin ait créé un rendez-vous avec l'UC1104. 
Un envoi de notification est programmé à une certaine heure avant l'heure du rendez-vous selon les 
paramètres utilisateur du patient.

\section{Pré-conditions}

Le patient doit avoir un rendez-vous de prévu.

\section{Post-conditions}

Aucune.

\section{Point d'inclusions et d'extensions}

SO. (Sans Objets).
\section{Règles de gestion}

\subsection{Classes concernées}

La classe Rendez-vous et Compte Patient sont concernées en lecture seulement.

\subsection{Validation des encodages de données}

Aucune donnée n'est transmise de cette manière au système.

\subsection{Validation des données en sortie}

Aucune donnée transmise par le système ne nécessite une telle vérification.

\section{Diagramme d'activité}

Ce UC ne possède pas de diagramme d'activité, il est du ressort du système d'agir en fonction
des choix de l'utilisateur.

\section{Interface utilisateur}

Etant un UC automatisé, cet UC ne possède pas d'interface graphique.


