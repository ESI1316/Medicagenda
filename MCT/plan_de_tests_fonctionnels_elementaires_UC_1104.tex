\chapter{Plan de test fonctionnels élémentaires \texttt{U.C. 1104}}
\section{Introduction}
\subsection{Objectifs du document}
Ce document présente  le plan des tests fonctionnels du \texttt{U.C. 1104} -
définir un rendez-vous. 
\subsection{Domaine de définition du document}
Ce document ne présente que les scénarios correspondant à la définition d’un
rendez-vous (correspondant au diagramme d’activités de la description fonctionnelle
du UC).
\subsection{Définitions, acronymes et abréviations}
Définition INAMI : voir étude de cas et MCD\footnote{\href{../MCD/MCD.pdf}{voir document M.C.D.}}.
\subsection{Références}
\begin{itemize}
	\item[] Étude de cas de l'agenda
		médical\footnote{\href{../Enonce_Travail_Synthese_14-15.pdf}{voir
		étude de cas}}
	\item[] M.C.D. de \texttt{Medicagenda}\footnote{\href{../MCD/MCD.pdf}{voir document M.C.D.}}
	\item[] M.C.T. de \texttt{Medicagenda}\footnote{\href{./MCT.pdf}{voir document M.C.T.}}
	\item[] Spécifications fonctionnelles du \texttt{U.C.
		1104}\footnote{\href{./specifications_fonctionnelles_UC_1104_definir_un_rendez-vous.pdf}{Voir
			spécifications fonctionnelles du \texttt{U.C. 1104}}}
	\end{itemize}

	\paragraph{Remarque}
	dans le plan de test, il est prévu de faire appel aux \texttt{U.C. 1101a/b} 
	\textbf{enregistrer un compte patient/médecin} ainsi qu'à l'\texttt{U.C. 1103}
	\textbf{définir une disponibilité}.
	Les \texttt{U.C. 1101a/b} et \texttt{1103} devront
	donc être réalisé et testé avant le présent \texttt{U.C. 1104}.

	\section{Types de tests élémentaires fonctionnels}
	\subsection{Validation des règles de saisie}
	\textbf{UC interactif}
	Validation des formats et valeurs des champs: voir description de l’interface et
	des messages d’erreur liés à son utilisation.

	La date et heure du rendez-vous à réserver ne doivent pas être antérieurs au
	moment où a lieu la réservation. 
	\subsection{Validation des règles de calcul}
	\texttt{S.O.}
	\subsection{Validation des mises à jour du SI}
	Vérification de la création correcte des nouveaux enregistrements dans la base
	de données, pour les différents scénarios décrits ci-dessous, ainsi que des
	liens entre objets de classes différentes. 
	Cela implique que le nouveau rendez-vous créé concerne les bons comptes
	utilisateurs, est à la date et heure convenue et que la plage horaire occupée
	par celui-ci soit dorénavant indisponible.
	\subsection{Validation des outputs utilisateurs}
	Vérification que les messages d’erreurs et de validations apparaissent
	pertinemment et s’affichent correctement. Les messages de récapitulation de la
	réservation doivent être pertinents pour les deux types d’utilisateurs (médecin
	et utilisateur) et exempts de toutes erreurs.
	\subsection{Tests de consolidation}
	Tous les scénarios peuvent commencer par un encodage de données à 
	vérifier par l’interface.;
	\subsection{Tests de saisie}
	À faire si tout nouveau développement d’interface.
	\subsection{Cas extrêmes}
	Au cours de différents scénarios, tests de l'abandon du \texttt{U.C.} et vérification que
	la base de données est effectivement restée intacte.
	\section{Scénarios de tests}
	Scénario |   cas  |         description    |   Résultat attendu 

	\begin{itemize}
		\item[] \texttt{scénario 0. cas 1.} 
			\begin{itemize}
				\item description: \\
					Le patient possède déjà un compte, a sélectionné un médecin ou une date ou une
					plage horaire mais désire annuler son choix pour en sélectionner un/une autre.
				\item résultat:\\
					La base de donnée n’est jamais modifiée, et la fenêtre de l’étape précédente
					est réaffichée.\\
			\end{itemize}
		\item[] \texttt{scénario 1. cas 1.1}
			\begin{itemize}
				\item description:  \\
					Le patient possède déjà un compte, a sélectionné et validé
					un médecin et une date et heure disponible.
				\item résultat: \\
					un fenêtre affiche le calendrier et la barre de recherche
					pour entrer le nom du medecin. 
					Une fois ce dernier et une date valide séléctionnés, est
					affichée une grille horaire de cette date avec les
					disponibilités du médecin. Après que la plage horraire eu
					été séléctionnée, 
					le message récapitulatif de la réservation s’est affiché et
					a été validé. 
					Le nouveau rendez-vous est enregistré, il s’affiche dans le
					calendrier du patient et dans celui du médecin, la
					disponibilité correspondant à la plage horaire du nouveau
					rendez-vous n’est plus.\\
			\end{itemize}
		\item[] \texttt{scénario 1. cas 1.2}
			\begin{itemize}
				\item description:  \\
					idem que 1.1 mais le patient n’est pas enregistré
				\item résultat: \\
					idem que 1.1 mais le patient s’ enregistre au préalable
					(appel au \texttt{U.C. 1101})\\
			\end{itemize}
		\item[] \texttt{scénario 2. cas 2.1}
			\begin{itemize}
				\item description:\\
					Le patient sélectionne un médecin et essaye de sélectionner
					une date ne contenant pas de disponibilité.
				\item résultat: \\
					Statu quo, les dates concernées doivent êtres non cliquable
					et rien ne doit alors se passer.\\
			\end{itemize}
		\item[] \texttt{scénario 2. cas 2.2}
			\begin{itemize}
				\item description:\\
					Le patient sélectionne un médecin et essaye de sélectionner
					une date antérieure au moment de la réservation.
				\item résultat: \\
					Idem que 2.1.\\
			\end{itemize}
		\item[] \texttt{scénario 2. cas 2.3}
			\begin{itemize}
				\item description:\\
					Le patient sélectionne un médecin et essaye de sélectionner
					une date à laquelle le médecin possède des disponibilités,
					mais une plage horaire à laquelle il possède déjà un
					rendez-vous.
				\item résultat: \\
					statu quo, les cases représentant ces disponibilités
					doivent être rougie avec un message décrivant le fait que le
					patient a déjà un rendez vous à ce moment là et donc,
					doivent êtres non cliquable et rien ne doit alors se passer.\\
			\end{itemize}
	\end{itemize}
	\newpage
	\section{Valeurs}
	\subsection{Valeurs d'initialisation}
	Pour pouvoir faire les tests des scénarios 1 et 2, il faudra disposer d’une
	base de données contenant un nombre minimal de comtes médecins avec
	disponibilités, ainsi que de quelques comptes patients.
	Il faudra également disposer du \texttt{U.C. 1101} testé qui peut être appelé par le
	présent \texttt{U.C. 1104}.
	\subsection{Valeurs spécifiques}
	Avant les tests, il faut constituer la base de données de tests et des cas
	de tests avec des disponibilités et rendez-vous déjà présents dans les
	calendrier des médecins et patients (ne pas oublier que l'environnement de
	tests (comme l'environnement de développement) doit être séparé de
	l'environnement de production).
	Les données spécifiques se retrouveront dans la base de données qui sera
	mise à la disposition des testeurs. Pour ``peupler'' cette base de données, il
	sera fait appel à des outils d’encodage rapide de données.
